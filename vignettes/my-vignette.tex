\documentclass[]{article}
\usepackage{lmodern}
\usepackage{amssymb,amsmath}
\usepackage{ifxetex,ifluatex}
\usepackage{fixltx2e} % provides \textsubscript
\ifnum 0\ifxetex 1\fi\ifluatex 1\fi=0 % if pdftex
  \usepackage[T1]{fontenc}
  \usepackage[utf8]{inputenc}
\else % if luatex or xelatex
  \ifxetex
    \usepackage{mathspec}
  \else
    \usepackage{fontspec}
  \fi
  \defaultfontfeatures{Ligatures=TeX,Scale=MatchLowercase}
\fi
% use upquote if available, for straight quotes in verbatim environments
\IfFileExists{upquote.sty}{\usepackage{upquote}}{}
% use microtype if available
\IfFileExists{microtype.sty}{%
\usepackage{microtype}
\UseMicrotypeSet[protrusion]{basicmath} % disable protrusion for tt fonts
}{}
\usepackage[margin=1in]{geometry}
\usepackage{hyperref}
\hypersetup{unicode=true,
            pdftitle={Accurate genetic and environmental covariance estimation with composite likelihood in genome-wide association studies},
            pdfauthor={Boran Gao},
            pdfborder={0 0 0},
            breaklinks=true}
\urlstyle{same}  % don't use monospace font for urls
\usepackage{color}
\usepackage{fancyvrb}
\newcommand{\VerbBar}{|}
\newcommand{\VERB}{\Verb[commandchars=\\\{\}]}
\DefineVerbatimEnvironment{Highlighting}{Verbatim}{commandchars=\\\{\}}
% Add ',fontsize=\small' for more characters per line
\usepackage{framed}
\definecolor{shadecolor}{RGB}{248,248,248}
\newenvironment{Shaded}{\begin{snugshade}}{\end{snugshade}}
\newcommand{\KeywordTok}[1]{\textcolor[rgb]{0.13,0.29,0.53}{\textbf{#1}}}
\newcommand{\DataTypeTok}[1]{\textcolor[rgb]{0.13,0.29,0.53}{#1}}
\newcommand{\DecValTok}[1]{\textcolor[rgb]{0.00,0.00,0.81}{#1}}
\newcommand{\BaseNTok}[1]{\textcolor[rgb]{0.00,0.00,0.81}{#1}}
\newcommand{\FloatTok}[1]{\textcolor[rgb]{0.00,0.00,0.81}{#1}}
\newcommand{\ConstantTok}[1]{\textcolor[rgb]{0.00,0.00,0.00}{#1}}
\newcommand{\CharTok}[1]{\textcolor[rgb]{0.31,0.60,0.02}{#1}}
\newcommand{\SpecialCharTok}[1]{\textcolor[rgb]{0.00,0.00,0.00}{#1}}
\newcommand{\StringTok}[1]{\textcolor[rgb]{0.31,0.60,0.02}{#1}}
\newcommand{\VerbatimStringTok}[1]{\textcolor[rgb]{0.31,0.60,0.02}{#1}}
\newcommand{\SpecialStringTok}[1]{\textcolor[rgb]{0.31,0.60,0.02}{#1}}
\newcommand{\ImportTok}[1]{#1}
\newcommand{\CommentTok}[1]{\textcolor[rgb]{0.56,0.35,0.01}{\textit{#1}}}
\newcommand{\DocumentationTok}[1]{\textcolor[rgb]{0.56,0.35,0.01}{\textbf{\textit{#1}}}}
\newcommand{\AnnotationTok}[1]{\textcolor[rgb]{0.56,0.35,0.01}{\textbf{\textit{#1}}}}
\newcommand{\CommentVarTok}[1]{\textcolor[rgb]{0.56,0.35,0.01}{\textbf{\textit{#1}}}}
\newcommand{\OtherTok}[1]{\textcolor[rgb]{0.56,0.35,0.01}{#1}}
\newcommand{\FunctionTok}[1]{\textcolor[rgb]{0.00,0.00,0.00}{#1}}
\newcommand{\VariableTok}[1]{\textcolor[rgb]{0.00,0.00,0.00}{#1}}
\newcommand{\ControlFlowTok}[1]{\textcolor[rgb]{0.13,0.29,0.53}{\textbf{#1}}}
\newcommand{\OperatorTok}[1]{\textcolor[rgb]{0.81,0.36,0.00}{\textbf{#1}}}
\newcommand{\BuiltInTok}[1]{#1}
\newcommand{\ExtensionTok}[1]{#1}
\newcommand{\PreprocessorTok}[1]{\textcolor[rgb]{0.56,0.35,0.01}{\textit{#1}}}
\newcommand{\AttributeTok}[1]{\textcolor[rgb]{0.77,0.63,0.00}{#1}}
\newcommand{\RegionMarkerTok}[1]{#1}
\newcommand{\InformationTok}[1]{\textcolor[rgb]{0.56,0.35,0.01}{\textbf{\textit{#1}}}}
\newcommand{\WarningTok}[1]{\textcolor[rgb]{0.56,0.35,0.01}{\textbf{\textit{#1}}}}
\newcommand{\AlertTok}[1]{\textcolor[rgb]{0.94,0.16,0.16}{#1}}
\newcommand{\ErrorTok}[1]{\textcolor[rgb]{0.64,0.00,0.00}{\textbf{#1}}}
\newcommand{\NormalTok}[1]{#1}
\usepackage{graphicx,grffile}
\makeatletter
\def\maxwidth{\ifdim\Gin@nat@width>\linewidth\linewidth\else\Gin@nat@width\fi}
\def\maxheight{\ifdim\Gin@nat@height>\textheight\textheight\else\Gin@nat@height\fi}
\makeatother
% Scale images if necessary, so that they will not overflow the page
% margins by default, and it is still possible to overwrite the defaults
% using explicit options in \includegraphics[width, height, ...]{}
\setkeys{Gin}{width=\maxwidth,height=\maxheight,keepaspectratio}
\IfFileExists{parskip.sty}{%
\usepackage{parskip}
}{% else
\setlength{\parindent}{0pt}
\setlength{\parskip}{6pt plus 2pt minus 1pt}
}
\setlength{\emergencystretch}{3em}  % prevent overfull lines
\providecommand{\tightlist}{%
  \setlength{\itemsep}{0pt}\setlength{\parskip}{0pt}}
\setcounter{secnumdepth}{0}
% Redefines (sub)paragraphs to behave more like sections
\ifx\paragraph\undefined\else
\let\oldparagraph\paragraph
\renewcommand{\paragraph}[1]{\oldparagraph{#1}\mbox{}}
\fi
\ifx\subparagraph\undefined\else
\let\oldsubparagraph\subparagraph
\renewcommand{\subparagraph}[1]{\oldsubparagraph{#1}\mbox{}}
\fi

%%% Use protect on footnotes to avoid problems with footnotes in titles
\let\rmarkdownfootnote\footnote%
\def\footnote{\protect\rmarkdownfootnote}

%%% Change title format to be more compact
\usepackage{titling}

% Create subtitle command for use in maketitle
\providecommand{\subtitle}[1]{
  \posttitle{
    \begin{center}\large#1\end{center}
    }
}

\setlength{\droptitle}{-2em}

  \title{Accurate genetic and environmental covariance estimation with composite
likelihood in genome-wide association studies}
    \pretitle{\vspace{\droptitle}\centering\huge}
  \posttitle{\par}
    \author{Boran Gao}
    \preauthor{\centering\large\emph}
  \postauthor{\par}
      \predate{\centering\large\emph}
  \postdate{\par}
    \date{2020-06-15}


\begin{document}
\maketitle

\subsubsection{Introduction}\label{introduction}

This vignette provides an introduction to the \texttt{GECKO} package. R
package \texttt{GECKO} implements GECKO, an accurate genetic and
environmental covariance estimation with composite likelihood in
genome-wide association studies. The package can be installed with the
command:

\texttt{library(devtools)}

\texttt{install\_github("borangao/GECKO")}

The package can be loaded with the command:

\begin{Shaded}
\begin{Highlighting}[]
\KeywordTok{library}\NormalTok{(}\StringTok{"GECKO"}\NormalTok{)}
\end{Highlighting}
\end{Shaded}

\subsubsection{Fit GECKO using example data within the
package}\label{fit-gecko-using-example-data-within-the-package}

We first load summary statistics and LD score.sumstat\_1 and sumstat\_2
are the files of the summary statistics with column name chr, bp, SNP,
A1, A2, N, Z, P representing chromosome, base pair position, SNP iD,
major allele, minor allele, number of individuals in the study, Z score,
P value. ldscore is the LD score file generated by LDSC software using
1000 genome reference panel. The example data and LD score could be
accessed using the code below.

\begin{Shaded}
\begin{Highlighting}[]
\KeywordTok{data}\NormalTok{(sumstat_}\DecValTok{1}\NormalTok{);}
\KeywordTok{data}\NormalTok{(sumstat_}\DecValTok{2}\NormalTok{);}
\KeywordTok{data}\NormalTok{(ldscore);}
\end{Highlighting}
\end{Shaded}

There are 6 arguments needed to be specified by GECKO including number
of observation of the both studies, number of overlapped samples,
whether to use weighted composite-lieklihood (usually is specified TRUE
to increase estimation accuracy), whether fix environmental covariance
to be zero if it's known that there is no overlapped sample, whether
testing for the genetic and envrionmental covariance (It takes more time
to test for the significant genetic and envrionmental
covariance).GECKO\_R is the function to calculate genetic and
environmental covariance. The example code of analysis is listed
below.Then we fit the GECKO using code below:

\begin{Shaded}
\begin{Highlighting}[]
\NormalTok{n1in<-}\KeywordTok{round}\NormalTok{(}\KeywordTok{mean}\NormalTok{(sumstat_}\DecValTok{1}\OperatorTok{$}\NormalTok{N))}
\NormalTok{n2in<-}\KeywordTok{round}\NormalTok{(}\KeywordTok{mean}\NormalTok{(sumstat_}\DecValTok{2}\OperatorTok{$}\NormalTok{N))}
\NormalTok{nsin<-}\DecValTok{0}
\NormalTok{Weightin =}\StringTok{ }\NormalTok{T }\CommentTok{#is always set for GECKO to improve the efficiency}
\NormalTok{Fix_Vein =}\StringTok{ }\NormalTok{T }\CommentTok{#(if two studies have non-overlapped samples, otherwise false)}
\NormalTok{Test =}\StringTok{ }\NormalTok{T }\CommentTok{#Test for significance or not}
\NormalTok{###Need to specify the number of individuals within each study: n1,n2, and number of the overlapping individuals in the two studies:nsin}
\NormalTok{###if the two samples are from the separate studies, nsin = 0, and Fix_Vein = 1}
\NormalTok{###if the two samples are from the same study, nsin need to be specified}

\NormalTok{Result<-}\KeywordTok{GECKO_R}\NormalTok{(sumstat_}\DecValTok{1}\NormalTok{,sumstat_}\DecValTok{2}\NormalTok{,n1in,n2in,nsin,ldscore,Weightin,Fix_Vein,Test)}
\CommentTok{#> 123456789101112131415161718192021222324252627282930313233343536373839404142434445464748495051525354555657585960616263646566676869707172737475767778798081828384858687888990919293949596979899100}
\end{Highlighting}
\end{Shaded}


\end{document}
